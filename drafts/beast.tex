\documentclass{acm_proc_article-sp}
\usepackage{svg}
\begin{document}

\title{
BEAST TITLE TO-FILL
%\titlenote{(Does NOT produce the permission block, copyright information nor page numbering). For use with ACM\_PROC\_ARTICLE-SP.CLS. Supported by ACM.}
}
\numberofauthors{5}
\author{
\alignauthor
Yang Shichu\\
       \affaddr{School of Cyber Science and Engineering}\\
       \affaddr{Huazhong University of Science and Technology}\\
       \email{sigeryeung@gmail.com}
\alignauthor
Shu Yi\\
       \affaddr{TO-FILL}\\
       \affaddr{TO-FILL}\\
       \email{TO-FILL}
\alignauthor
Su Haochen\\
       \affaddr{TO-FILL}\\
       \affaddr{TO-FILL}\\
       \email{TO-FILL}
\and
\alignauthor
Li Yucong\\
       \affaddr{TO-FILL}\\
       \affaddr{TO-FILL}\\
       \email{TO-FILL}
\alignauthor
Liao Haicheng\\
       \affaddr{TO-FILL}\\
       \affaddr{TO-FILL}\\
       \email{TO-FILL}
}
\date{27 July 2021}
\maketitle
\begin{abstract}
Transport Layer Security (TLS) is an protocol that provides communication
security over networks. However, there is a flaw in TLS 1.0 where the initial
vectors for block ciphers are predictable. The BEAST attack, with some
prerequisites and efforts, allows attackers in the middle to decrypt those
encrypted messages.
This paper will demonstrate the procedures of the BEAST attack, and propose
methods in simulation and vulnerability detection.
\end{abstract}

% A category with the (minimum) three required fields
% \category{H.4}{Information Systems Applications}{Miscellaneous}
%A category including the fourth, optional field follows...
% \category{D.2.8}{Software Engineering}{Metrics}[complexity measures, performance measures]

% \terms{}

\keywords{BEAST attack, TLS flaws, CBC exploits, vulnerability detection} % NOT required for Proceedings

\section{Introduction}
Transport Layer Security (TLS) has several versions. The specification for TLS
1.0 is RFC 2246\cite{rfc2246}.

%\end{document}  % This is where a 'short' article might terminate
\section{Background}
\subsection{A glance at TLS}
\subsection{CBC in block ciphers}
CBC is one of the modes of operation used in block ciphers.

Supposing that $P_1,P_2,\cdots P_n$ are the plaintext blocks, with a initial vector $IV$, we have:

$$
\begin{aligned}
C_1&=E_k(P_1\oplus IV)\\
C_i&=E_k(P_{i}\oplus C_{i-1}) (i\geq 2)
\end{aligned}
$$

to obtain ciphertext blocks $C_1,C_2,\cdots,C_n$.

\begin{figure}[htb]
  \centering
  \includesvg[keepaspectratio, width=\linewidth]{./figures/cbc-encryptor.drawio.svg}
  \caption{CBC encryptor}
\end{figure}
\section{Threat Model}

\section{Demonstration}
\section{Practicality and Defense}
\subsection{Praticality}
While BEAST attacks are theoretically feasible, but through

% \subsection{}
%ACKNOWLEDGMENTS are optional

% The following two commands are all you need in the
% initial runs of your .tex file to
% produce the bibliography for the citations in your paper.
\bibliographystyle{abbrv}
\bibliography{beast}  % sigproc.bib is the name of the Bibliography in this case
% You must have a proper ".bib" file
%  and remember to run:
% latex bibtex latex latex
% to resolve all references
%
% ACM needs 'a single self-contained file'!
%
%APPENDICES are optional
\balancecolumns
\appendix
%Appendix A
\section{Headings in Appendices}
\balancecolumns
\end{document}
